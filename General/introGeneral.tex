
\begin{center}
	\Large\textbf{Introduction Général}
\end{center}

Les réseaux de capteurs sans fil sont récemment devenus un sujet de recherche de premier plan. L’évolution technologique de ces derniers a permis une production de masse peu coûteuse de nœuds de capteurs autonomes, qui, malgré leurs miniaturisation, ont des capacités de détection, de traitement et de communication particulièrement avancées [53], et ont un fort potentiel économique à long terme. Flexibilité, hautes capacités de captage, coût réduit, installation rapide sont les caractéristiques qui ont permis aux réseaux de capteurs d’avoir des nouveaux domaines d’applications multiples et excitants. Ce large étendu d’application fera de cette technologie émergente une partie intégrale de nos vies futures.
\subparagraph{}
Cependant, la réalisation des réseaux de capteurs nécessite la satisfaction de certaines contraintes qui découlent d’un nombre de facteurs guidant la phase de conception, tel que la tolérance aux pannes, la scalabilité, le coût et la consommation de l’énergie qui est la substance de notre sujet.
Le problème majeur des réseaux de capteurs sans fil est le problème des trous d’énergie. Les nœuds les plus proches de la région de puits mourront plus tôt des sous-régions externes car ils envoient leurs propres données et transmettent également les données des sous-régions externes au récepteur. Donc, après très peu de temps, un trou d’énergie vient près de la région de l’évier. Après cela, les données ne peuvent plus être transmises, même s'il reste encore de l'énergie dans les nœuds de la région externe, ce qui affecte la durée de vie du réseau. Par conséquent, pour permettre d’augmenter la  longévité, d’un nœud capteur nous nous intéressons particulièrement à la résolution d’un problème d’optimisation combinatoire, qui est relativement nouveau dans les RCSFs, intitulé le problème de l’arbre dominant ou DTP.  Ce dernier offre une ossature pour le routage des données à travers le réseau, du fait que les informations de routage seront stockés uniquement au niveau des nœuds de l’arbre dominant. 
\subparagraph{}
La résolution de ce problème qui est NP-Difficile [13] [14], fait recours aux méthodes d’optimisation combinatoire, à savoir, les méta-heuristiques, tout en tirant profit des propriétés de la théorie des graphes. Ensuite, des essais d'amélioration sont apportés en travaillant sur ses paramètres afin d'obtenir les méthodes les plus efficaces possibles. La qualité de chaque méthode est évaluée en la comparant à d'autres méthodes proposées pour le problème étudié. Malheureusement, d'après les No Free Lunch Theorems [Wolpert 1997], il n'existe pas de métaheuristique qui soit meilleure que toutes les autres métaheuristiques pour tous les problèmes. Dans la pratique, il existera toujours des instances pour lesquelles une métaheuristique est meilleure qu'une autre. Quelque soit la métaheuristique choisie, elle présente des avantages et des inconvénients. C’est la raison pour laquelle la méthode de collaboration a été proposée. Cette méthode propose de regrouper un ensemble d'heuristiques ou métaheuristiques et d'établir un mécanisme pour identifier et sélectionner les méthodes de recherche les plus efficaces au cours du processus d'optimisation. Généralement, certaines études visent à produire des heuristiques constructives qui construiront une solution, étape par étape. Les heuristiques sont utilisées pour décider comment étendre une solution partielle. Ces méthodes ont tendance à être rapide. D'autres études visent à produire des heuristiques d'amélioration ou de perturbation travaillant sur une solution candidate déterminée et essayant d'améliorer sa qualité. Ces méthodes sont plus lentes mais fournissent les meilleurs résultats finaux. 

\subparagraph{}
Notre mémoire s’articule autour de cinq chapitres. Le premier présente un état de l’art des Réseaux de Capteurs Sans Fils (RCSF). L’étude des méthodes d’optimisation combinatoires et de résolution exactes et approchées fait l’objet du chapitre deux. Le troisième chapitre sera consacré à l’étude du problème de l’arbre dominant ainsi que des différents travaux effectués pour sa résolution et sur lesquels repose notre travail. Dans le quatrième chapitre, nous présentons l’approche de résolution développée . Quant au chapitre cinq, il est consacré à l’expérimentation ainsi qu’à l’analyse des
performances. Nous terminons par une conclusion générale et des perspectives.
