\cleardoublepage

\addcontentsline{toc}{chapter}{\numberline{}3eme chapiter}
\addtocontents{lof}{\textbf{Chapter A3}}

\setcounter{chapter}{3}
\setcounter{section}{0}
\setcounter{figure}{0}

\begin{center}
	\Huge\textbf{3eme chapter}
\end{center}

\section{Introduction}


\section{Définition du problème}



\section{Complexité et approximation}

\cleardoublepage


%% 1

\begin{algorithm}[H]
\label{alg3:RSDT}
\caption{Recuit simulé}
\KwData{un graphe général pondéré non orienté $G=(V, E, w)$}
\KwResult{Un DT de $G$ }
\SetAlgoLined
\DontPrintSemicolon

Initialiser une liste pour sauvegarder le DT et son poids \;
Trouver un sommet factice u en G avec un degré minimum \;
Transformer $G$ en $G’ = (V’, E’, w’)$ en utilisant la technique de transformation\;
\For{chaque nœud $v$ où $(u, v) \in E $}{
	Exécution de l'algorithme DST \cite{charikar1999approximation} dans G et définir v comme racine r pour obtenir un DT \;
	Sauvegarder le DT et son poids dans la liste \;	
}
 Retourner le DT avec poids minimum dans la liste \;
\end{algorithm}


\cleardoublepage

%% 2

\begin{algorithm}[H]
\label{alg3:prim}
\caption{Prim}
\KwData{un graphe général pondéré non orienté $G=(V, E, w)$}
\KwResult{Un arbre couvrant de poids minimal T}
\SetAlgoLined
\DontPrintSemicolon

\tcp{Initialisation de T}
sommets $\gets$ un sommet de $G$ qu’on choisit \;
arêtes $\gets$ aucune 

\Repeat{tous les sommets de G soient dans T}{
Trouver toutes les arêtes de $G$ qui relient un sommet de T et un sommet extérieur à T \;
Sélectionner l’arête ayant le plus petit poids parmis l’ensemble des arêtes trouvées \;
Ajouter à T cette arête et le sommet correspondant \;
}
Retourner T \;
\end{algorithm}


\cleardoublepage

\begin{algorithm}[H]
\label{alg3:HDT}
\caption{ pseudo-code H\_DT}
\KwData{un graphe général pondéré non orienté $G=(V, E, w)$}
\KwResult{sortie: un arbre dominant $DT \subseteq E$ }
\SetAlgoLined
\DontPrintSemicolon

\tcp{Initialise DT}
$DT \gets \varnothing $  \;

\For{chaque sommet $i \in V$}{
	Mark$[i] \gets $ 0 \;	
}

Calcule les chemins les plus courts entre toutes les paires de sommets de G \;
Triez toutes les arêtes de E en ordre non décroissant en fonction de leur poids \;

\While{tous les sommets de V ne sont pas marqués}{
Sélectionnez une arête non sélectionnée $e_{ij}$ avec un coût minimal, dont au moins un point d'extrémité n'est pas marqué \;

\For{chaque sommet v adjacent à  $i$ ou $j$ }{
	Mark$[v] \gets$ 1 \;
}

\eIf{ DT = $ \varnothing $ ou $e_{ij}$ est adjacent à une arête dans DT}{
	$ DT \gets DT \cup \{ e_{ij} \} $ \;
}{
	Trouver le chemin le plus court ST reliant DT et $\{i, j\}$ \;
	$ DT \gets DT \cup ST \cup \{ e_{ij} \}$ \;
}

\For{chaque sommet v en ST}{
	$ Mark[v] \gets 1$ \;
	\For{chaque sommet k adjacent à v }{
		$ Mark[k] \gets 1$ \;
	}
}

}

Appliquer la procédure d'élagage sur DT \;
Reconnectez les sommets dominants de DT en y construisant un arbre couvrent de poids minimal \;
Appliquer l'élagage sur le DT \; 
Reconnectez les sommets dominants de DT en y construisant un arbre couvrent de poids minimal \;
Retourner DT \;
\end{algorithm}
