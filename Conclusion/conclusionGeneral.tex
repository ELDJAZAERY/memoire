
\addcontentsline{toc}{chapter}{\numberline{}Conclusion générale}

\begin{center}
	\LARGE\textbf{Conclusion générale}
\end{center}

%\thispagestyle{empty}
\thispagestyle{plain}
%\renewcommand{\headrulewidth}{0pt}

Les travaux présentés dans ce projet de fin d’études traitent le fameux problème de l’arbre dominant qui est un des problèmes les plus traités dans le domaine de l'optimisation combinatoire.

Ce problème est apparu à l’époque où les Réseaux de Capteurs Sans Fils (RCSF) ont connu un  grand développement ce qui leur a permis d’occuper une place très importante dans le monde de la recherche scientifique.  En effet, il est né lors de de l’ apparition de la contrainte d’énergie  qui représente un inconvénient majeur dans les RCSFs. Les plus grands taux de la consommation d’énergie ont lieu lors de la transmission des données.

Une des solutions consiste à fournir une ossature pour l’optimisation du routage des données permettant ainsi une économie d’énergie. Ceci revient à rechercher un arbre dominant de poids minimal.

L’objectif principal de notre projet de fin d’études est de proposer une méthode pour la résolution du problème de l’arbre dominant. Afin d’arriver à bout de notre solution, une étude des différentes techniques d’optimisation combinatoire a été nécessaire, ainsi qu’une étude approfondie des travaux élaborés dans le même but. Ce qui nous a permis  d'adapter quelques méta-heuristiques inspirées de travaux antécédents.

Nous avons également mis en œuvre une nouvelle approche nommé l’approche coopérative, qui consiste à faire coopérer plusieurs maté-heuristiques esclaves soumises au contrôle d'une méta-heuristique maître. Malgré la difficulté du problème traité, ce fut véritablement une expérience fructueuse qui nous a ouvert les portes vers de nouveaux horizons. Pour conclure, nous avons réussi à mener notre projet à terme. L’application réalisée, nous a permis, comme prouvé dans les différents graphes d’atteindre des résultats très proches de ceux des travaux antérieurs et d’améliorer la majorité. 

En guise de perspectives, nous envisageons d’intégrer un mécanisme d’adaptation des paramètres qui permettra de guider le comportement du processus de recherche en fonction d’indicateurs collectés au cours de l’exécution, ce qui ouvre par la suite une nouvelle fenêtre pour la recherche. Nous aimerions aussi, adapter l’approche proposée à d’autres problèmes similaires.
